\batchmode
\makeatletter
\def\input@path{{\string"/cygdrive/c/Users/Dennis/Documents/Atmel Studio/6.2/coinForth/doc/ARD101/\string"/}}
\makeatother
\documentclass[10pt,english]{article}
\usepackage[T1]{fontenc}
\usepackage[latin1]{inputenc}
\usepackage{listings}
\usepackage[dvips]{geometry}
\geometry{verbose,lmargin=2cm,rmargin=2cm}
\usepackage{fancyhdr}
\pagestyle{fancy}
\setlength{\parskip}{\smallskipamount}
\setlength{\parindent}{0pt}
\usepackage{babel}
\usepackage{longtable}
\usepackage{textcomp}
\usepackage{url}
\usepackage{makeidx}
\makeindex
\usepackage[dvips]{graphicx}
\usepackage[unicode=true,pdfusetitle,
 bookmarks=true,bookmarksnumbered=false,bookmarksopen=false,
 breaklinks=false,pdfborder={0 0 1},backref=false,colorlinks=false]
 {hyperref}

\makeatletter

%%%%%%%%%%%%%%%%%%%%%%%%%%%%%% LyX specific LaTeX commands.
%% Because html converters don't know tabularnewline
\providecommand{\tabularnewline}{\\}

%%%%%%%%%%%%%%%%%%%%%%%%%%%%%% Textclass specific LaTeX commands.
\usepackage{noweb}

%%%%%%%%%%%%%%%%%%%%%%%%%%%%%% User specified LaTeX commands.
%Document generated by wvWare/wvWare version 1.2.4
%wvWare written by Caolan.McNamara@ul.ie
%http://wvware.sourceforge.net
\@ifundefined{definecolor}
 {\usepackage{color}}{}
\usepackage[dvips]{graphics}\usepackage{longtable}\usepackage{times}
\usepackage{comment}\usepackage{lastpage}
\usepackage{microtype}\DisableLigatures[-]{}

\usepackage[normalem]{ulem}\newcommand{\suppress}[1]{}\newcommand{\deleted}[1]{\xout{#1}}\newcommand{\revised}[1]{\uline{#1}}\newlength\wvtextpercent\setlength{\wvtextpercent}{0.009\textwidth}

\newbox\strikebox\def\strike#1{\setbox\strikebox \hbox{<#1>}\hbox{\raise0.5ex\hbox to 0pt{\vrule height 0.4pt width \wd\strikebox\hss}\copy\strikebox}}

\lhead{Public Domain} \chead{} \rhead{\jobname}
\lfoot{Public Domain} \cfoot{} \rfoot{Page \thepage\ of \pageref{LastPage}}

\includecomment{Foo}\excludecomment{Bar}

\newcommand\projectVersion{0.0.0.2}
\newcommand\projectDate{04/14/14}

\begin{Foo}
\newcommand\projectName{ARD101}
\end{Foo}

\begin{Bar}
\newcommand\projectName{Bar portion of FooBar}
\end{Bar}



\makeatother

\begin{document}

\title{\textbf{ARD101 Tutorial Conversion}}

\maketitle
\noindent \begin{center}
Public Domain
\par\end{center}

\begin{longtable}{cc}
\hline 
\multicolumn{1}{c}{\textbf{Software Version}} &
\textbf{Date}\tabularnewline
\multicolumn{1}{c}{\projectVersion} &
\projectDate\tabularnewline
\hline 
\end{longtable}

\tableofcontents{}




\section{OSEPP 101 Arduino Basics Starter Kit}

This \projectName\ project started when Fry's put this kit on sale
for \$35.99.

\url{http://osepp.com/products/kits/101-arduino-basic-starter-kit/}

It contains an UNO R3 Plus processor, which is compatible with, at
least, Dr. Ting's 328eForth for Arduino.

\url{http://www.offete.com/328eForth.html}

It is also compatible with SwiftX AVR and probably with amForth, but
I haven't gotten the latter working yet.

\url{http://www.forth.com/downloads/SwiftX-Eval-AVR-3.7.1-f4qbm8hnnrg5r42ko.exe}

\url{http://amforth.sourceforge.net/}

The Kit also contains all the parts needed to complete 7 different
tutorials involving flashing LEDs, playing tones and reading voltages
and GPIOs, which can provide a very basic comparison between the C
and Forth programming languages. Unfortunately, while the Kit is complete
for programming in C, you still need an additional AVR ISP programmer
to work with Forth. While Atmel does sell one that is very mature,
I decided to try to find one that was a little smaller. After a couple
of misfires, I went back to the one recommended by Leon Wagner from
Forth, Inc. at the February 2011 SVFIG meeting:

\url{http://www.forth.org/svfig/kk/02-2011.html}

\url{http://www.pololu.com/product/1300/}


\subsection{arduino-ble-dev-kit}

And now, in the pursuit of the Internet of Things (IoT), I am moving
this to Bluetooth LE:

\url{http://blog.onlycoin.com/posts/?category=Open+Source}

They did not pre-burn the boot loaders or software with the boards
produced in a 2nd run, as this service was not offered by their contract
manufacturer in China. However, do not fear! They have a very simple
fix that has taken most people 10 minutes or less to get you up and
programming with your awesome dev kit, just follow these steps...

\url{http://ross-arduinoprojects.blogspot.com/2014/04/setting-up-coin-ble-dev-kit.html}

Well, any implication that this would be an easy transition has proven
to be misleading, but I did finally get the SwiftX distress sample
to run last night. The Pololu Programmer doesn't work, because this
is a 3.3 volt system running at 8 MHz. So, I have had to get the Atmel
AVRISP mkII and the TI CC-DEBUGGER for the Bluetooth module. The \texttt{\textbf{CPUCLOCK}}
change in SwiftX AVR was pretty easy to figure out, but I still haven't
figured out what's needed in eForth. I changed the baud rate value
in \texttt{\textbf{STOIO}}, but I still don't get any output. I guess
I should take another stab at amForth.

The AVRISP mkII proved to be problematic too. Atmel switched to a
Jungo interface in their Studio 5, but I had installed the \texttt{\textbf{avrispmkii
libusb win32 1.2.1.0}} which prevented the Jungo interface from attaching
to the mkII. Installing Atmel Studio 6.2.1153 .net fixed that problem
and using it to recompile eForth fixed that too. I had a point in
time that I had amForth working, but that proved to be fleeting and
doesn't work today. They appear to have issues with uploading code
that I haven't figure out yet too (See section 3.7.1 of the AmForth
Documentation, Release 5.3).


\subsubsection{BLE Sample}

There are many Bluetooth Low Energy (BLE) samples around right now.
It's the latest fad for the IoT and Texas Instruments CC2540 is one
of the more popular chips supporting the protocol. The CC2540 combines
an excellent RF transceiver with an industry-standard enhanced 8051
MCU. In \ref{fig:BLE Sample} you can see that the chip is alive and
transmitting, but I still have to figure out how to use it. Until
then, I just leave this running on 3 AAA batteries to see if I can
even make a dent in the battery voltage. Coin predicts a couple of
years for their product, so I may have to wait a very long time before
I see any change. We'll see!

\begin{figure}
\caption{\label{fig:BLE Sample}\protect\includegraphics[height=0.7\textheight]{0_cygdrive_c_Users_Dennis_Documents_Atmel_Studio_6_2_coinForth_doc_ARD101_BLE_Sample.png}}


\end{figure}



\subsection{Atmel Studio 6}

The Kit does require it's own software. ARDUINO 1.0.5-r2 - 2014.01.08
is the latest version, but I ended up using ARDUINO 1.5.5-r2 BETA
2014.01.10 to support my Windows 8.1 computer. Now, I see that they
are up to version 1.6.3, which also works.

\url{http://arduino.cc/en/Main/Software}

While SwiftX is self contained, amForth and eForth require a separate
assembler and compiler system. The most recent is Atmel's Studio 6
(Version 6.1.2730 - Service Pack 2).

\url{http://www.atmel.com/tools/ATMELSTUDIO.aspx}

Getting all of this setup is a much more significant part of the problem
than it needs to be, but then again, supporting all of the updates
of all of the pieces is certainly an expensive proposition. I just
wish more vendors considered it to be a priority. OSEPP is doing a
better job than most, but even they are at the mercy of the Arduino
open-source community, and Win8.1 is causing most companies issues
with their driver security requirements.

\url{http://preview.tinyurl.com/krnp7nv}

Atmel has also gone through many variations to its compiler suite,
and with this latest version, they chose to use Microsoft's Visual
Studio as the IDE for their compiler. They are still stuck in the
VS2010 version, but at least they have a reasonable update mechanism
and they do provide the older version for the less adventurous. I
personally believe in continuous integration, so I still am struggling
to figure out how to get amForth to work. Dr. Ting's eForth was not
too difficult, but the Pololu programmer instructions have not been
updated to the latest menus in Studio 6. The instructions that say
to ``select Add STK500\ldots{} from the Tools menu'' should actually
be to select Add target... and Select the STK500 tool. A minor but
frustratingly significant difference.

After having to rebuild my computer and do a quick contract between
jobs, I've gotten this setup again, and found the command line syntax
to flash without loading the entire IDE. It is:

\begin{lstlisting}
atprogram -t stk500 -c 6 -i ISP -d ATmega328P -v program -f target.hex
\end{lstlisting}


You may need to adjust some of the parameters for your environment.

Also after an even longer hiatus, an item that I had to go find again,
was how to connect the AVRISP MkII to the arduino-ble-dev-kit. I found:

\url{http://www.atmel.com/webdoc/avrispmkii/avrispmkii.section.zgf_vsd_lc.html}
and

\url{http://img.onlycoin.com/arduino-ble-dev-kit/pinout.png} which
solved that dilemma.

\begin{figure}


\caption{\protect\includegraphics[width=0.7\textwidth]{1_cygdrive_c_Users_Dennis_Documents_Atmel_Studio_6_2_coinForth_doc_ARD101_coin.jpg}}
\end{figure}



\subsubsection{328eForth v2.20}

Once I finally got everything setup properly, I was able to backup
the existing flash image, which is essential if you want to return
to the Kit's original Tutorials. However, I lost that backup when
I had to rebuild my computer, so I had to find out where the ``official''
image is called \texttt{\textbf{\small ATmegaBOOT\_168\_atmega328\_pro\_8MHz.hex}}
and was at:

\texttt{\textbf{\small C:\textbackslash{}Arduino\textbackslash{}arduino-1.5.5-r2\textbackslash{}hardware\textbackslash{}arduino\textbackslash{}avr\textbackslash{}bootloaders\textbackslash{}atmega\textbackslash{}}}{\small \par}

Dr. Ting's instructions also say to set the High Fuse byte to 0xD8,
which I have so far, found to be unnecessary and the original setting
of 0xDE (BOOTSZ = 256W\_3F00) works the same. Once I flashed the \texttt{\textbf{328eforth.hex}}
and connected the serial port to PuTTY, I got the following:

\begin{lstlisting}
328eForth v2.20
ok
words
 VARIABLE CONSTANT CREATE IMMEDIATE : ] ; OVERT ." $" ABORT" WHILE ELSE AFT THEN 
REPEAT IF AGAIN UNTIL NEXT FOR BEGIN LITERAL COMPILE [COMPILE] , IALLOT ALLOT 
D- D+ D> > 2- 2+ 1- 1+ READ WRITE ERASE COLD WORDS .S IDUMP DUMP ' QUIT EVAL [ 
QUERY EXPECT NAME> WORD CHAR \ ( .( ? . U. U.R .R CR ITYPE TYPE SPACES SPACE KEY 
NUMBER? DECIMAL HEX #> SIGN #S # HOLD <# FILL CMOVE @EXECUTE TIB PAD HERE ICOUNT 
COUNT +! PICK DEPTH */ */MOD M* * UM* / MOD /MOD M/MOD UM/MOD WITHIN MIN MAX < 
U< = ABS - DNEGATE NEGATE INVERT + 2DUP 2DROP ROT ?DUP BL 2/ 2* LAST DP CP 
CONTEXT HLD 'EVAL 'TIB #TIB >IN SPAN TMP BASE 'BOOT UM+ XOR OR AND 0< OVER 
SWAP DUP DROP >R R@ R> C@ C! FLUSH I! IC@ I@ @ ! EXIT EXECUTE EMIT ?KEYok 
ok
\end{lstlisting}


So now, I can start translating the tutorials.


\subsubsection{amforth 5.8 ATmega328P Forthduino}

I have a \texttt{\textbf{C:\textbackslash{}amforth}} folder with versions
5.1, 5.3, 5.4, and now 5.8 in it, so I have been trying to do this
for a while now. I finally noticed:

\begin{lstlisting}
.equ F_CPU = 16000000 -> 8000000
\end{lstlisting}


and:

\begin{lstlisting}
.set BAUD = 38400 
\end{lstlisting}


and thought I might try PuTTY at 19200 baud and pressed the reset
button multiple times, until I got:

\begin{lstlisting}
amforth 5.8 ATmega328P Forthduino
> 
\end{lstlisting}


Not rocket science, and doesn't work or always take keyboard input,
but encouraging. On the other hand:

\begin{lstlisting}
Assembly failed, 39 errors, 44 warnings
\end{lstlisting}


And how to make:

\begin{lstlisting}
Atmel Studio 6 (Version: 6.2.1563 - Service Pack 2)  2014 Atmel Corp. All rights reserved.
\end{lstlisting}


Find the source files...

In the Project -> Properties -> Toolchain -> General, check Generate
EEP file and

\begin{lstlisting}
Include Paths (-I)
../../amforth-5.8/common
../../amforth-5.8/avr8
\end{lstlisting}


Other optimization flags:

\begin{lstlisting}
-v0
\end{lstlisting}


and finally!

\begin{lstlisting}
------ Build started: Project: coinForth, Configuration: Debug AVR ------
Build started.
Project "coinForth.asmproj" (default targets):
Target "PreBuildEvent" skipped, due to false condition;
('$(PreBuildEvent)'!='') was evaluated as (''!='').
Target "CoreBuild" in file "C:\Program Files (x86)\Atmel\Atmel Studio 6.2\Vs\Assembler.targets"
from project "C:\Users\Dennis\Documents\Atmel Studio\6.2\coinForth\coinForth\coinForth.asmproj"
(target "Build" depends on it):
        Task "RunAssemblerTask"
C:\Program Files (x86)\Atmel\Atmel Toolchain\AVR Assembler\Native\2.1.1175\avrassembler\avrasm2.exe
 -fI -o "coinForth.hex"  -m "coinForth.map"  -l "coinForth.lss"  -S "coinForth.tmp"
 -W+ie -I"../../amforth-5.8/common" -I"../../amforth-5.8/avr8"  -e uno.eep -v0 -im328Pdef.inc
 -d "C:\Users\Dennis\Documents\Atmel Studio\6.2\coinForth\coinForth\Debug/coinForth.obj"
 "C:\Users\Dennis\Documents\Atmel Studio\6.2\coinForth\coinForth\coinForth.asm"
 -I "C:\Program Files (x86)\Atmel\Atmel Toolchain\AVR Assembler\Native\2.1.1175\avrassembler\Include"
        Done executing task "RunAssemblerTask".
Done building target "CoreBuild" in project "coinForth.asmproj".
Target "PostBuildEvent" skipped, due to false condition;
('$(PostBuildEvent)' != '') was evaluated as ('' != '').
Target "Build" in file "C:\Program Files (x86)\Atmel\Atmel Studio 6.2\Vs\Avr.common.targets"
from project "C:\Users\Dennis\Documents\Atmel Studio\6.2\coinForth\coinForth\coinForth.asmproj"
(entry point):
Done building target "Build" in project "coinForth.asmproj".
Done building project "coinForth.asmproj".

Build succeeded.
========== Build: 1 succeeded or up-to-date, 0 failed, 0 skipped ========== 
\end{lstlisting}


I had to change the project name from my\_amforth to coinForth because
amforth.asm opened my\_amforth.asm, which makes the build fail.

I also had to turn off the \texttt{\textbf{EESAVE}} and \texttt{\textbf{BOOTRST}}
fuses and change the \texttt{\textbf{BOOSTSZ}} fuse from \texttt{\textbf{1024W\_3C00}}
to \texttt{\textbf{2048W\_3800}}. (E.g. HIGH = 0xD9).

I found that amForth can not handle direct file input, but the 

\begin{lstlisting}
C:\Users\Dennis\Documents\Atmel Studio\6.2\coinForth\amforth-5.8\tools>python amforth-shell.py
|I=appl_defs: 0 loaded 
|I=Entering amforth interactive interpreter 
|I=getting MCU name.. 
|I=successfully loaded register definitions for atmega328p 
|I=getting filenames on the host 
|I=  Reading C:\Users\Dennis\Documents\Atmel Studio\6.2\coinForth\amforth-5.8\avr8\devices\atmega328p
|I=  Reading C:\Users\Dennis\Documents\Atmel Studio\6.2\coinForth\amforth-5.8\avr8\lib 
|I=  Reading . 
|I=getting filenames from the controller 
(ATmega328P)> 5 .
5  ok 
(ATmega328P)> words 
-1 2 1 = 2literal s>d spaces space cr bounds ?stack tolower toupper turnkey bl hex decimal bin allot here ehere dp key? key emit? emit pad #tib >in tuck 2drop 2 dup cells base state f_cpu fill s" ." words show-wordlist within max min mod / negate u/mod u.r u. d= (defer) defer@ defer! Udefer! Udefer@ Rdefer! Rdefer@ Edefer! Edefer@ throw catch handler warm cold pause quit int-trap int@ int! -int +in t cell+ pick abs /mod dinvert d- d+ d2/ d2* nr> n>r @i (!i-nrww) !i @e !e 2r> 2> r 1ms up! up@ >< cmove> unloop i sp! sp@ rp! rp@ +! rshift lshift 1- 1+ xor or and 2* 2/ invert um* um/mod m* + - log2 > < u> u< 0 true d0< d0> 0> 0< 0= <> r@ >r r> nip rot drop over swap ?dup dup !u @u c@ c! ! @ (value) execute exit applturnkey i-cell+ postpone (marker) end-code code get-recognizers set-recognizers set-order forth-wordlist only wordlist set-current nfa>lfa compare get-order get-current map-stack set-stack get-stack ?abort abort abort" [char] immediate recurse user constant variable [ ] ; :noname : does> latest reveal wlscope header create lp lp0 >l l> endloop ?do leave +loop loop do again until repeat while begin then else if ahead sliteral literal ['] , compile ( \ (create) find-name nfa>cfa name>string traverse-wordlist search-wordlist r:fail r:word rec:word rec:num r:dnum r:num interpret do-recognizer depth rp0 sp sp0 parse-name /string source parse >number number char refill accept cscan cskip ' type icount itype s, digit? u d/mod ud.r ud. . d. .r d.r sign #> #s # <# hold hld environment init-user ee>ram source-tib refill-tib tib 2swap cmove dnegate dabs j * icompare to unused noop ver name>flags umin umax ud* m+ 1w.slot 1w.reset +usart ubrr tx?-poll tx-poll rx?-isr rx-isr  ok 
(ATmega328P)>
\end{lstlisting}


I haven't had a reason to use PowerShell yet, but this might be the
time

\begin{figure}
\caption{\protect\includegraphics[width=0.7\textwidth]{2_cygdrive_c_Users_Dennis_Documents_Atmel_Studio_6_2_coinForth_doc_ARD101_PowerShell.png}}
\end{figure}


\url{http://druffer.github.io/coinForth/}


\subsection{SwiftX for AVR}

\url{http://www.forth.com/embedded/eval-upgrade.html?MCU=AVR}

Initially, I was stuck in the SwiftX AVR Target Reference Manual,
Appendix A.1.1 Uno Board Overview. I was trying to get the RELOAD!
command to work, but it would not work with the Pololu USB AVR Programmer.
Once I got it working as an STK500 in Atmel Studio, and read further
in Appendix D: Atmel STK500, I saw that this is the ``normal'' way
to use this interface. So, now I can start including that system in
the translation too.

\begin{lstlisting}
SwiftX Evaluation AVR 3.7.0 01-Jan-2014
INCLUDE DEBUG
   Start      End      Size     Used   Unused    Type   Name
    0000     7FFF     32768     7308    25460    CDATA  FLASH
    0100     01FF       256       29      227    IDATA  IRAM
    0200     08FF      1792      421     1371    UDATA  URAM
TARGET READY
SwiftX/AVR Arduino Uno SOS Demo  ok
2 6 + . 8  ok
go
TARGET READY  ok
\end{lstlisting}


Unfortunately, my latest attempt is giving me this:

\begin{lstlisting}
C:\ForthInc\Projects\ARD101>atprogram -t stk500 -c 11 -i ISP -d ATmega328P -v program -f target.hex
[DEBUG] Starting execution of "program"
[DEBUG] Starting process 'C:\Program Files (x86)\Atmel\Atmel Studio 6.2\atbackend\atbackend.exe'
[DEBUG] Connecting to TCP:127.0.0.1:53208
[WARNING] Could not establish communication with the tool. (TCF Error code: 1)
[WARNING] Could not create tool context. Retrying, 3 more attempts...
[WARNING] Failed to open \\.\COM11. Error 0x5. (TCF Error code: 1)
[WARNING] Could not create tool context. Retrying, 2 more attempts...
[WARNING] Failed to open \\.\COM11. Error 0x5. (TCF Error code: 1)
[WARNING] Could not create tool context. Retrying, 1 more attempts...
[ERROR] Could not establish communication with the tool. (TCF Error code: 1) 
\end{lstlisting}


Still trying to figure that issue out.


\section{Tutorials}

Starting from OSEPP's learning center, I have shortened the URL for
each of the subsections below to fit on a printed page.

\url{http://osepp.com/learning-centre/start-here/101-basic-starter-kit/}

\url{http://tinyurl.com/megobz3}

I have also setup the hardware interfaces so that all 7 of the tutorials
are connected at the same time. The only overlap that this creates
is with the 7 segment LED. This just means that the speaker clicks
during the LED tutorial, but otherwise, all of the I/O used in the
tutorials functions properly

\begin{figure}
\caption{\protect\includegraphics[width=0.7\textwidth]{3_cygdrive_c_Users_Dennis_Documents_Atmel_Studio_6_2_coinForth_doc_ARD101_ARD101.jpg}}
\end{figure}



\subsection{Tutorial 1: Loading the First Sketch}

\url{http://tinyurl.com/megobz3/tutorial-1-loading-the-first-sketch/}

This tutorial looks to be well represented in the \texttt{\textbf{flasher.txt}}
sample that is included with eForth. However, 1st you need to have
a terminal emulator that can send text files with a 900 ms delay in
between lines. 900 ms is probably way too slow, but the system can
not handle no delay between lines. 200 ms seems about right. Unfortunately,
PuTTY can not do this. Realterm is a reasonable substitute, but scrollback
is an issue.

\url{http://realterm.sourceforge.net/}

I've always preferred the capabilities of HyperACCESS, which is the
parent of Windows old HyperTerminal. It's expensive, but I've seen
issues with just about any other terminal program and I don't recall
ever finding an issue with HyerACCESS. They also still sell the original
HyperTerminal Private Edition if you want something less expensive.

\url{http://www.hilgraeve.com/hyperaccess-trial/}

With decent scroll back capabilities, I was able to see that \texttt{\textbf{flasher.txt}}
required \texttt{\textbf{io-core.txt}} which was not loading properly.
Eventually, I figured out that I needed to load \texttt{\textbf{marker.txt}}
1st. Even Dr. Ting's documentation has a mistake there in that \texttt{\textbf{hello-world.txt}}
also requires \texttt{\textbf{marker.txt}}, but in the end, the system
can finally be considered to be functional.

\begin{lstlisting}
1000 3 manyok
\end{lstlisting}


Note that the ``ok'' doesn't output a space first, so the acknowledgment
can be a little confusing.


\subsubsection{Tutorial1}

\nwfilename{ARD101.nw}\nwbegincode{1}\sublabel{NWARD9-TutD-1}\nwmargintag{{\nwtagstyle{}\subpageref{NWARD9-TutD-1}}}\moddef{Tutorial1.ino~{\nwtagstyle{}\subpageref{NWARD9-TutD-1}}}\endmoddef
/* Tutorial 1
Blink
Turns on an LED on for one second, then off for one second, repeatedly.
This example code is in the public domain.
*/
// Pin 13 has an LED connected on most Arduino boards.
// give it a name:
int led = 13;
// the setup routine runs once when you press reset:
void setup() \{              
// initialize the digital pin as an output.
pinMode(led, OUTPUT);    
\}
// the loop routine runs over and over again forever:
void loop() \{
digitalWrite(led, HIGH);   // turn the LED on (HIGH is the voltage level)
delay(1000);               // wait for a second
digitalWrite(led, LOW);    // turn the LED off by making the voltage LOW
delay(1000);               // wait for a second
\}
\nwnotused{Tutorial1.ino}\nwendcode{}\nwbegindocs{2}\nwdocspar


\subsubsection{flasher.txt}

William F. Ragsdale had written these demo applications for Arduino
with AmForth. Dr. Ting modified them so that they work properly under
328eForth.\index{flasher.txt}

\nwenddocs{}\nwbegincode{3}\sublabel{NWARD9-flaB-1}\nwmargintag{{\nwtagstyle{}\subpageref{NWARD9-flaB-1}}}\moddef{flasher.txt~{\nwtagstyle{}\subpageref{NWARD9-flaB-1}}}\endmoddef
\\ FLASHER.txt to Demo LED control              WFR 2011-01-27 
( must have io-core.txt installed )
chop-flasher
marker chop-flasher  ( a forget point)
$23 value PortB  $26 value PortC  $29 value PortD
  5 value LED
: 1-cycle  ( ms_delay ---   flash LED on then off )
    PortB LED PoBiHi   dup ms   PortB LED PoBiLo   ms ;
: many  ( on_time flashes ---   produce controlled LED flashes)
    PortB LED PoBiOut ( set LED pin as output)
    for aft  dup 1-cycle then next drop ;
( use 'many' leading with on-time and # of flashes )
( end of flasher.txt )
flush
\nwnotused{flasher.txt}\nwendcode{}\nwbegindocs{4}\nwdocspar

Note that in both C and Forth, there are many support routines that
are not always listed. Knowing the environment is always key to your
productivity and you can usually learn a lot by examining the sample
source listings. You should also notice that the \texttt{\textbf{setup}}
and \texttt{\textbf{loop}} functions need to be done explicitly in
Forth. You can always assume that you need to do those steps, but
how often and in what order is typically, an application specific
requirement. Thus the use of parameters is more typical in Forth than
infinite loops.

You should also notice that while the C routines deal with sequential
pins which span multiple ports, the Forth routines deal with the port
bits directly. The onboard LED is on pin 13, but it can also be referenced
as bit 5 on Port B. The lower 8 pins are on Port D.


\subsubsection{flasher.frt}

In amforth-5.8, the original flasher that was mentioned above is no
longer found, so I need to create it.

\nwenddocs{}\nwbegincode{5}\sublabel{NWARD9-flaB.2-1}\nwmargintag{{\nwtagstyle{}\subpageref{NWARD9-flaB.2-1}}}\moddef{flasher.frt~{\nwtagstyle{}\subpageref{NWARD9-flaB.2-1}}}\endmoddef
\\ flasher.frt to Demo LED control
0 constant false  #include io-core.f
\\ requires: in application master file
   .set WANT_PORTB = 1
   .set WANT_PORTC = 1
   .set WANT_PORTD = 1
   .set WANT_TIMER_COUNTER_0 = 1
   .set WANT_SPI = 1
#include timer0.frt  #include timer.frt

$23 value PortB  $26 value PortC  $29 value PortD
  5 value LED
: 1-cycle  ( ms_delay ---   flash LED on then off )
    PortB LED PoBiHi   dup ms   PortB LED PoBiLo   ms ;
: many  ( on_time flashes ---   produce controlled LED flashes)
    PortB LED PoBiOut ( set LED pin as output)
    for aft  dup 1-cycle then next drop ;
( use 'many' leading with on-time and # of flashes )
\nwnotused{flasher.frt}\nwendcode{}\nwbegindocs{6}\nwdocspar


\subsubsection{distress.f}

An onboard LED example is also included in SwiftX AVR, but it is listed
as Copyright 2001-2007 FORTH, Inc. You should look at it and execute
it to make sure everything is working properly. However, I will not
list it here to avoid any copyright infringement. It is interesting
to note that this example puts the \texttt{\textbf{SOS}} distress
code into a background \texttt{\textbf{BEACON}} task, which allows
it to continue running while you continue to exercise the tether interface.
This is extremely useful, but for compatibility, I will not use it
for the rest of these tutorials. Instead, I will attempt to use the
same code on all of the Forth systems.

However, there will be some differences, and I will need a modified
version of the io-core.f support from eForth:\index{io-core.f}

\nwenddocs{}\nwbegincode{7}\sublabel{NWARD9-io*9-1}\nwmargintag{{\nwtagstyle{}\subpageref{NWARD9-io*9-1}}}\moddef{io-core.f~{\nwtagstyle{}\subpageref{NWARD9-io*9-1}}}\endmoddef
\\ Port Input Output for AmForth                  DaR 28Mar14
\\ loaded as  io-core.f
\\ Modified for 328eForth, 23mar11cht
\\ Modified for SwiftX, daruffer@gmail.com

\\ manually begin with chop-io entered

: mask ( bit# --- port_mask  convert bit to 8 bit mask)
     1 swap lshift  ;

: DDR  \\ port --- port' adjust input port# to DDR
    1- ;

: Input \\ port --- port adjust input port# to output
    2 - ;

: RegFrom \\ Reg mask --- value  read masked bits from register
        \\ To read all bits:  PortB true RegFrom -> value
     swap c@  and ;

: RegTo \\ Reg mask new ---  write masked new into register
     over and >r   invert over c@ and   r> or  swap c! ;

: PoBiI/O  \\ port bit direction --- configure bit in/out
    rot DDR   rot mask   rot RegTo 

: PoBiOut  \\ port bit --- configure as output
    true  PoBiI/O ;

: PoBiRead  \\ port bit --- value  read bit value from port
     swap Input swap mask RegFrom ;

: PoBiHi  \\ port bit --- set port bit 0..7 high
     mask true  RegTo ;

: PoBiLo   \\ port bit --- clear port bit 0..7 low
     mask false RegTo ;

: PoBiIn   \\ port bit --- configure as input,  no pull-up
    2dup false PoBiI/O   PoBiLo ( pullup inactive) ;

: PoBiInPu \\ port bit --- configure as input with pull-up
    2dup false PoBiI/O   PoBiHi ( pullup active) ;

\\ read bits from register  Reg#         select      RegFrom
\\ write bits to register   Reg#         select bits RegTo
\\ write 1-bit to register  Reg#   5 mask true       RegTo
\\ write 0-bit to register  Reg#   5 mask false      RegTo
\\ configure bits as output PortB DDR    select True RegTo
\\ write bits to output     PortB Output select bits RegTo
\\ configure bit as output  PortB     LED            PoBiOut
\\ bit as input with pullup PortB     Switch3        PoBiInPu
\\ read bit from port       PortB     Switch3        PoBiRead
\\ write 1-bit to port      PortB     LED            PoBiHi
\\ write 0-bit to port      PortB     LED            PoBiLo
\\ Note, when initializing a 16 bit register, TCNT1 etc. it
\\  must be written directly hi/lo not using RegTo.
\\  The proper form to clear is:    TCNT1hi false c!
\\                         then:    TCNT1lo false c!
\\                                  

\\ end of io-core.txt
\nwnotused{io-core.f}\nwendcode{}\nwbegindocs{8}\nwdocspar

Note that the differences are:
\begin{enumerate}
\item The \texttt{\textbf{marker}} concepts don't really apply here.
\item The port names refer to the output port, rather than the input port.
Thus, the adjustments are reversed.
\item The \texttt{\textbf{flush}} concept used by eForth also doesn't apply
here.
\end{enumerate}

\subsection{Tutorial 2: Controlling Digital Outputs}

\url{http://tinyurl.com/megobz3/tutorial-2-controlling-digital-outputs/}

\begin{lstlisting}
500 3 cyclesok
\end{lstlisting}



\subsubsection{Tutorial2}

\nwenddocs{}\nwbegincode{9}\sublabel{NWARD9-TutD.2-1}\nwmargintag{{\nwtagstyle{}\subpageref{NWARD9-TutD.2-1}}}\moddef{Tutorial2.ino~{\nwtagstyle{}\subpageref{NWARD9-TutD.2-1}}}\endmoddef
/*
Tutorial 2 Digital Output
*/
int LED0  = 2;     // Use digital pin 2 to drive the white LED
int LED1  = 3;     // Use digital pin 3 to drive the yellow LED
int LED2  = 4;     // Use digital pin 4 to drive the green LED
int LED3  = 5;     // Use digital pin 5 to drive the red LED
void setup() \{
  // initialize digital pin 2 to 5 as output:
  pinMode(LED0, OUTPUT);  
  pinMode(LED1, OUTPUT);  
  pinMode(LED2, OUTPUT);  
  pinMode(LED3, OUTPUT);  
\}
void loop()\{
  // Toggle each LED at a time with a 500ms delay
  digitalWrite(LED0, HIGH);
  delay(500);
  digitalWrite(LED0, LOW);
  delay(500);
  digitalWrite(LED1, HIGH);
  delay(500);
  digitalWrite(LED1, LOW);
  delay(500); 
  digitalWrite(LED2, HIGH);
  delay(500);
  digitalWrite(LED2, LOW);
  delay(500); 
  digitalWrite(LED3, HIGH);
  delay(500);
  digitalWrite(LED3, LOW);
  delay(500); 
\}
\nwnotused{Tutorial2.ino}\nwendcode{}\nwbegindocs{10}\nwdocspar


\subsubsection{cycle.txt\index{cycle.txt}}

\nwenddocs{}\nwbegincode{11}\sublabel{NWARD9-cyc9-1}\nwmargintag{{\nwtagstyle{}\subpageref{NWARD9-cyc9-1}}}\moddef{cycle.txt~{\nwtagstyle{}\subpageref{NWARD9-cyc9-1}}}\endmoddef
\\ cycle.txt to Demo multiple LED control      DaR 2014-02-16
chop-cycle
marker chop-cycle
\\ Define LED port bits and flashing order
CREATE LEDS   4 2* 1 + allot \\ Number of LEDS, then order
: \\LEDS ( --- Initialize the RAM array )   4 LEDS C!
   LEDS count 2* PortD fill \\ Overfill the Port addresses to save code
   2 LEDS 1 + C!  3 LEDS 3 + C!  4 LEDS 5 + C!  5 LEDS 7 + C! ;
: cycle ( time port bit --- flash LED on then off )
   2dup PoBiHi  rot dup ms  rot rot PoBiLo  ms ;
: cycles ( time cycles --- produce cycles of LED flashes )   \\LEDS
   LEDS count for aft  count >r count r> PoBiOut  then next drop
   for aft  LEDS count for aft  count >r count >r over r> r>
         cycle  then next  drop  then next  drop ;
flush
\nwnotused{cycle.txt}\nwendcode{}\nwbegindocs{12}\nwdocspar

Note a few principles here:
\begin{enumerate}
\item Look for patterns of doing things 3 or more times and factory them
out. Each LED name was used 3 times, which leads to putting them into
an array, which only needs to be referenced twice. This also allows
a significant reduction in code size.
\item Be careful where things are compiled when systems have multiple address
spaces. I had thought that \texttt{\textbf{,}} would work to create
the table, but no, I had to resort to a much less elegant solution.
\item Still, the lack of elegance is at compile time and does not effect
the run time behavior. That makes it much less objectionable.
\item Last minute uglyness is the requirement for a \texttt{\textbf{flush}}
before \texttt{\textbf{LEDS}} can be referenced. Otherwise, the system
would reboot while compiling this code. That's the risk for compile
time initialization and thus, why it is now in a definition and called
everytime \texttt{\textbf{cycles}} starts up. This gives it some runtime
overhead, but saves compatibility issues with other systems.
\item Don't be afraid of passing multiple parameters. Up to 3 parameters
are easily handled in Forth and even more can be handled with minimal
difficulties. Watch for literals or fixed values that might change
over time, like the LED parameters here in the \texttt{\textbf{cycle}}
routine. The original \texttt{\textbf{1-cycle}} routine could have
been written this way with some forethought.
\item Know when to stop factoring things out. I could have broken \texttt{\textbf{cycles}}
down into, at least, 2 other words. However, again you should remember
the rule of 3. I might use a similar pattern 1 more time in the next
tutorial, but as with most test code, a 3rd time is unlikely.
\end{enumerate}

\subsubsection{cycle.f\index{cycle.f}}

\nwenddocs{}\nwbegincode{13}\sublabel{NWARD9-cyc7-1}\nwmargintag{{\nwtagstyle{}\subpageref{NWARD9-cyc7-1}}}\moddef{cycle.f~{\nwtagstyle{}\subpageref{NWARD9-cyc7-1}}}\endmoddef
\\ cycle.f to Demo multiple LED control      DaR 2014-03-29
\\ Define LED port bits and flashing order
CREATE LEDS   4 2* 1 + allot \\ Number of LEDS, then order
: \\LEDS ( --- Initialize the RAM array )   4 LEDS C!
   LEDS count 2* PortD fill \\ Overfill the Port addresses to save code
   2 LEDS 1 + C!  3 LEDS 3 + C!  4 LEDS 5 + C!  5 LEDS 7 + C! ;
: cycle ( time port bit --- flash LED on then off )
   2dup PoBiHi  rot dup ms  rot rot PoBiLo  ms ;
: cycles ( time cycles --- produce cycles of LED flashes )   \\LEDS
   LEDS count 0 do  count >r count r> PoBiOut  loop  drop
   0 do  LEDS count 0 do  count >r count >r over r> r>
         cycle  loop  drop  loop  drop ;
\nwnotused{cycle.f}\nwendcode{}\nwbegindocs{14}\nwdocspar

Note that the differences are:
\begin{enumerate}
\item The \texttt{\textbf{marker}} and \texttt{\textbf{flush}} concepts
don't really apply here.
\item The \texttt{\textbf{for aft ... then next}} structure is directly
replaced with the more standard \texttt{\textbf{0 do ... loop}} structure.
\end{enumerate}

\subsection{Tutorial 3: Using Digital Input}

\url{http://tinyurl.com/megobz3/tutorial-3-using-digital-input/}


\subsubsection{Tutorial3}

\nwenddocs{}\nwbegincode{15}\sublabel{NWARD9-TutD.3-1}\nwmargintag{{\nwtagstyle{}\subpageref{NWARD9-TutD.3-1}}}\moddef{Tutorial3.ino~{\nwtagstyle{}\subpageref{NWARD9-TutD.3-1}}}\endmoddef
/*
Tutorial 3 Digital Input
*/
const int buttonPin = 12;     // Use digital pin 12 for the button pin
int buttonState = 0;          // variable for storing the button status
void setup() \{
  // initialize the pushbutton pin as an input:
  pinMode(buttonPin, INPUT);
  // initialize the serial port;
  Serial.begin(9600);  // start serial for output 
\}
void loop()\{
  // read the state of the pushbutton value:
  buttonState = digitalRead(buttonPin);
  // Output button state
  Serial.print("The button state is ");
  Serial.println(buttonState);
  // Delay 1000ms
  delay(1000);
\}
\nwnotused{Tutorial3.ino}\nwendcode{}\nwbegindocs{16}\nwdocspar


\subsubsection{button.txt\index{button.txt}}

\nwenddocs{}\nwbegincode{17}\sublabel{NWARD9-butA-1}\nwmargintag{{\nwtagstyle{}\subpageref{NWARD9-butA-1}}}\moddef{button.txt~{\nwtagstyle{}\subpageref{NWARD9-butA-1}}}\endmoddef
\\ button.txt to Demo Digital input      DaR 2014-02-17
chop-button
marker chop-button
4 value buttonPin \\ Use digital pin 12 for the button pin
: states ( --- read state of button )
   PortB buttonPin 2dup PoBiIn  PoBiRead
   begin  PortB buttonPin PoBiRead  2dup - if
         cr ." The button state is "  dup . swap
   then  drop  ?key until  drop ;
flush
\nwnotused{button.txt}\nwendcode{}\nwbegindocs{18}\nwdocspar

A few more principles:
\begin{enumerate}
\item For testing words, like these, it is often convenient to just wait
for a key press to terminate the loop. You have an interactive terminal
loop running anyway. You might just as well use it. However, be warned
that eForth appears to have a bug with \texttt{\textbf{until}}. The
\texttt{\textbf{drop}}, or anything else after \texttt{\textbf{until}},
never executes. Not a big problem here, and I have reported it.
\item We also don't need to initialize the serial port because it is the
terminal loop. I suspect that this may not always be the case.
\item Don't add things that you don't use. Note that \texttt{\textbf{buttonState}}
is not needed in Forth, when the stack can hold the state.
\item Don't time a polled event if you don't need to. There's no need to
report the state unless it changes.
\end{enumerate}

\subsubsection{button.f\index{button.f}}

\nwenddocs{}\nwbegincode{19}\sublabel{NWARD9-but8-1}\nwmargintag{{\nwtagstyle{}\subpageref{NWARD9-but8-1}}}\moddef{button.f~{\nwtagstyle{}\subpageref{NWARD9-but8-1}}}\endmoddef
\\ button.f to Demo Digital input      DaR 2014-03-29
4 value buttonPin \\ Use digital pin 12 for the button pin
: states ( --- read state of button )
   PortB buttonPin 2dup PoBiIn  PoBiRead
   begin  PortB buttonPin PoBiRead
      2dup = while  nip  repeat
   cr ." The button state is " . drop ;
\nwnotused{button.f}\nwendcode{}\nwbegindocs{20}\nwdocspar

Note that the differences are:
\begin{enumerate}
\item The \texttt{\textbf{marker}} and \texttt{\textbf{flush}} concepts
don't really apply here.
\item Since I don't have serial port support in SwiftX, I only loop until
the button state changes.
\end{enumerate}

\subsection{Tutorial 4: An LED Game}

\url{http://tinyurl.com/megobz3/tutorial-4-an-led-game/}


\subsubsection{Tutorial4}

\nwenddocs{}\nwbegincode{21}\sublabel{NWARD9-TutD.4-1}\nwmargintag{{\nwtagstyle{}\subpageref{NWARD9-TutD.4-1}}}\moddef{Tutorial4.ino~{\nwtagstyle{}\subpageref{NWARD9-TutD.4-1}}}\endmoddef
/*
  Tutorial 4 Digital Input and Output Game
  In this game, the LED will loop from white, yellow, green, red
  then back to white.  The goal is to press the push button at the exact
  moment when the green LED is ON. Each time you got it right, the LED
  will speed up and the difficulty will increase.
*/
int currentLED = 2;
int delayValue = 200;
void setup() \{
  // initialize digital pin 12 as input;
  pinMode(12, INPUT);   // button input
  // initialize digital pin 2 to 5 as output:
  pinMode(2, OUTPUT);   // white LED
  pinMode(3, OUTPUT);   // yellow LED
  pinMode(4, OUTPUT);   // green LED
  pinMode(5, OUTPUT);   // red LED
\}
int checkInput() \{ 
  if (digitalRead(12) == 0) \{
    return 1;
  \} else \{
    return 0;
  \}
\}
void loop()\{
  // Check if the button is press at the right moment
  if (digitalRead(12) == 0) \{
    if (currentLED == 4) \{
       // Blink the correct (green) LED
       digitalWrite(4, HIGH);
       delay(200);
       digitalWrite(4, LOW);
       delay(200);
       digitalWrite(4, HIGH);
       delay(200);
       digitalWrite(4, LOW);
       delay(200);
       // Speed up the LEDs
       delayValue = delayValue - 20; 
    \} else \{
       // Blink the wrong LED
       digitalWrite(currentLED, HIGH);
       delay(200);
       digitalWrite(currentLED, LOW);
       delay(200);
       digitalWrite(currentLED, HIGH);
       delay(200);
       digitalWrite(currentLED, LOW);
       delay(200);
    \}
  \}
  // Loop LED from white > yellow > green > red
  digitalWrite(currentLED, HIGH);
  delay(delayValue);
  digitalWrite(currentLED, LOW);
  delay(delayValue);
  currentLED = currentLED + 1;
  if (currentLED > 5) \{
     currentLED = 2;
  \}
\}
\nwnotused{Tutorial4.ino}\nwendcode{}\nwbegindocs{22}\nwdocspar


\subsubsection{game.txt\index{game.txt}}

\nwenddocs{}\nwbegincode{23}\sublabel{NWARD9-gam8-1}\nwmargintag{{\nwtagstyle{}\subpageref{NWARD9-gam8-1}}}\moddef{game.txt~{\nwtagstyle{}\subpageref{NWARD9-gam8-1}}}\endmoddef
\\ game.txt Digital Input and Output Game      DaR 2014-02-20
chop-game
marker chop-game
variable delayValue
: game ( --- cycles LEDs and check button presses )   \\LEDS
   LEDS count for aft  count >r count r> PoBiOut  then next drop
   PortB buttonPin PoBiIn  200 delayValue !
   begin  LEDS count for aft  count >r count r>
         2dup delayValue @ rot rot cycle
         PortB buttonPin PoBiRead 0 = if
            rot dup LEDS count 1 - 2* + = if
               delayValue @ 20 - dup 0 = if
                  ." You win!" 2drop drop exit
               then  delayValue !
            then  rot rot
            2dup 200 rot rot cycle
            2dup 200 rot rot cycle
         then
         2drop  ?key if
            drop exit
         then
   then next  drop  again ;
flush 
\nwnotused{game.txt}\nwendcode{}\nwbegindocs{24}\nwdocspar

Things to note here:
\begin{enumerate}
\item A pointer can easily serve as an index. You just have to use something
a little less opaque than a number for comparison. Typically, that
comparison value can be computed, which is certainly a requirement
for using this technique.
\item The use of multiple \texttt{\textbf{exit}}s with an endless \texttt{\textbf{again}}
loop is common in Forth and not something that should be frowned upon
as it is with other languages.
\item Unfortunately, this technique appears to also have an issue, like
\texttt{\textbf{until}} does, as was discussed earlier. In this case,
the chip reboots as soon as a key is pressed, or when you win. In
the later case, the message doesn't even get a chance to finish.
\end{enumerate}

\subsubsection{game.f\index{game.f}}

\nwenddocs{}\nwbegincode{25}\sublabel{NWARD9-gam6-1}\nwmargintag{{\nwtagstyle{}\subpageref{NWARD9-gam6-1}}}\moddef{game.f~{\nwtagstyle{}\subpageref{NWARD9-gam6-1}}}\endmoddef
\\ game.f Digital Input and Output Game      DaR 2014-03-29
variable delayValue
: game ( --- cycles LEDs and check button presses )   \\LEDS
   LEDS count 0 do  count >r count r> PoBiOut  loop  drop
   PortB buttonPin PoBiIn  200 delayValue !
   begin  LEDS count 0 do  count >r count r>
         2dup delayValue @ rot rot cycle
         PortB buttonPin PoBiRead 0 = if
            rot dup LEDS count 1 - 2* + = if
               delayValue @ 20 - dup 0 = if
                  ." You win!" 2drop drop exit
               then  delayValue !
            then  rot rot
            2dup 200 rot rot cycle
            2dup 200 rot rot cycle
         then  2drop
      loop  drop
   again ;
\nwnotused{game.f}\nwendcode{}\nwbegindocs{26}\nwdocspar

Note that the differences are:
\begin{enumerate}
\item The \texttt{\textbf{marker}} and \texttt{\textbf{flush}} concepts
don't really apply here.
\item The \texttt{\textbf{for aft ... then next}} structure is directly
replaced with the more standard \texttt{\textbf{0 do ... loop}} structure.
\item Since I don't have serial port support in SwiftX, I loop until the
CPU is reset.
\item It appears that the eForth reboot also effects this system, but I'm
not sure how pervasive it is yet.
\end{enumerate}

\subsection{Tutorial 5: Building Voltage Meter}

\url{http://tinyurl.com/megobz3/tutorial-5-building-voltage-meter/}


\subsubsection{Tutorial5}

\nwenddocs{}\nwbegincode{27}\sublabel{NWARD9-TutD.5-1}\nwmargintag{{\nwtagstyle{}\subpageref{NWARD9-TutD.5-1}}}\moddef{Tutorial5.ino~{\nwtagstyle{}\subpageref{NWARD9-TutD.5-1}}}\endmoddef
/*
  Tutorial 5: Volt Meter
*/
int sensorPin = A0;    // select the analog input pin
int sensorValue = 0;   // variable to store the value coming from the sensor
float sensorVoltage = 0; // variable to store the voltage coming from the sensor
void setup() \{
  Serial.begin(9600);  // start serial for output
\}
void loop() \{
  // Read the value from the analog input pin
  // A value of 1023 = 5V, a value of 0 = 0V
  int sensorValue = analogRead(sensorPin);
  // Convert sensor value to voltage
  float sensorVoltage= sensorValue*(5.0/1023.0);
  // print sensor value
  Serial.print("The voltage is ");
  Serial.println(sensorVoltage);
  // delay by 1000 milliseconds:
  delay(1000);                 
\}
\nwnotused{Tutorial5.ino}\nwendcode{}\nwbegindocs{28}\nwdocspar


\subsubsection{volts.f\index{volts.f}}

\nwenddocs{}\nwbegincode{29}\sublabel{NWARD9-vol7-1}\nwmargintag{{\nwtagstyle{}\subpageref{NWARD9-vol7-1}}}\moddef{volts.f~{\nwtagstyle{}\subpageref{NWARD9-vol7-1}}}\endmoddef
\\ volts.f to Demo Analog input      DaR 2014-03-29
0 value analogPin \\ Use analog pin 0 for the voltage pin
: volts ( --- read state of pot )
   PortC analogPin PoBiInPu
   $40 ADMux c!  $C3 ADCSra c!  ADCL @
   cr ." The voltage is "  500 1023 */
   0 <# # # $2E hold # #> type space ;
\nwnotused{volts.f}\nwendcode{}\nwbegindocs{30}\nwdocspar

Notes for the SwiftX version only:
\begin{enumerate}
\item Although the eForth manual does have a section regarding analog inputs,
I chose to use the SwiftX model after losing my work in a computer
rebuild.
\item While the C version used floating point operators, this is often overkill
for embedded systems. Fixed point math allows you to use the faster
integer math operators and you simply need to keep track of where
the decimal point is. Typically, the only place where this information
is needed is when the value is displayed (see the \texttt{\textbf{\$2E
hold}} above). You do need to be concerned with the range of the value,
which is why the {*}/ operator uses a double-cell intermediate result.
\item Typically, you also would want to separate the voltage conversion
from the display output, but for simplicity, I did not do that here.
\end{enumerate}

\subsection{Tutorial 6: Using Buzzer to Play a Melody}

\url{http://tinyurl.com/megobz3/tutorial-6-using-buzzer-to-play-a-melody/}


\subsubsection{Tutorial6a}

\nwenddocs{}\nwbegincode{31}\sublabel{NWARD9-TutE-1}\nwmargintag{{\nwtagstyle{}\subpageref{NWARD9-TutE-1}}}\moddef{Tutorial6a.ino~{\nwtagstyle{}\subpageref{NWARD9-TutE-1}}}\endmoddef
/*   Tutorial 6a: Simple Scale Sweep */
int buzzerPin = 8;    // Using digital pin 8
#define NOTE_C6  1047
#define NOTE_D6  1175
#define NOTE_E6  1319
#define NOTE_F6  1397
#define NOTE_G6  1568
#define NOTE_A6  1760
#define NOTE_B6  1976
#define NOTE_C7  2093
void setup() \{
  // nothing to setup
\}
void loop() \{
  //tone(pin, frequency, duration)
  tone(buzzerPin, NOTE_C6, 500);
  delay(500);
  tone(buzzerPin, NOTE_D6, 500);
  delay(500);
  tone(buzzerPin, NOTE_E6, 500);
  delay(500);
  tone(buzzerPin, NOTE_F6, 500);
  delay(500);
  tone(buzzerPin, NOTE_G6, 500);
  delay(500);
  tone(buzzerPin, NOTE_A6, 500);
  delay(500);
  tone(buzzerPin, NOTE_B6, 500);
  delay(500);
  tone(buzzerPin, NOTE_C7, 500);
  delay(500);
\}
\nwnotused{Tutorial6a.ino}\nwendcode{}\nwbegindocs{32}\nwdocspar


\subsubsection{tone.f\index{tone.f}}

The eForth system ha this tone generation code, which uses pin 6 on
PortD, rather than pin 0 on PortB. So far, I have not figured out
how to generate a tone on PortB, so I have simply switched the buzzer
over to the bit used by eForth. This really is so much easier than
figuring out the code.

\nwenddocs{}\nwbegincode{33}\sublabel{NWARD9-ton6-1}\nwmargintag{{\nwtagstyle{}\subpageref{NWARD9-ton6-1}}}\moddef{tone.f~{\nwtagstyle{}\subpageref{NWARD9-ton6-1}}}\endmoddef
\\ Audio tone generator                      30Mar14
\\ Modified for 328eForth, 23mar11cht
\\ Modified for SwiftX by daruffer@gmail.com

\\ Must have io-core.f installed

6 value Tone-out \\ PortD bit 6

binary

: setup-osc \\ prescale limit  ---  limit 1..255, prescale 1..5
   PortD   Tone-out                  PoBiOut \\ setup output pin
   OCR0A   true      rot ( limit   ) RegTo
   TCCR0A  11000011  01000010        RegTo   \\ CTC mode
           00000101  min    0  max    \\ form TCCR0B prescale
   TCCR0B  00001111 rot ( prescale ) RegTo ;  \\ and tone on

: tone-off \\  ---  end output tone setting prescale to zeros
   TCCR0B  00000111  false RegTo ;

decimal

78 value Limit   4 value Prescale  \\ 400 Hz tone parameters

: ud/mod ( ud1 n -- rem ud2 ) >R 0 R@ UM/MOD R> SWAP >R UM/MOD R> ;
: Hertz   \\ frequency --- load Limit and Prescale
    $1200 $7A ( 8000000. ) rot ud/mod  \\ total scale as rem double-quot
    dup            if ( >16 bits)  1024  5 else
    over $C000 and if ( >14 bits)   256  4 else
    over $F800 and if ( >10 bits)    64  3 else
    over $FF00 and if ( > 8 bits)     8  2 else   1  1
                      then then then then
      to Prescale   um/mod to Limit  drop drop ( two remainders ) ;

: tone-on   \\ ---  begin tone from fixed presets
     Prescale  Limit  setup-osc ;

: note  \\ duration ---    generate timed tone for duration msec.
     tone-on  ms  tone-off ;

\\ End of tone.f
\nwnotused{tone.f}\nwendcode{}\nwbegindocs{34}\nwdocspar

Notes for the SwiftX version only:
\begin{enumerate}
\item The \texttt{\textbf{marker}} and \texttt{\textbf{flush}} concepts
don't really apply here.
\item The SwiftX system defines all of the ports as compile time \texttt{\textbf{EQU}}
constants, so I do not need to define them here.
\item I have not defined a way to compile the musical notes yet, but the
Ring Tone Text Transfer Language looks interesting.\\
See: \url{http://www.srtware.com/index.php?/ringtones/rtttlformat.php}
\end{enumerate}

\subsubsection{scale.f\index{scale.f}}

\nwenddocs{}\nwbegincode{35}\sublabel{NWARD9-sca7-1}\nwmargintag{{\nwtagstyle{}\subpageref{NWARD9-sca7-1}}}\moddef{scale.f~{\nwtagstyle{}\subpageref{NWARD9-sca7-1}}}\endmoddef
\\ Play musical scale                      13Apr14

CREATE scale \\ sequence of notes, pauses and times
   1047 , 500 , 0 , 500 , 1175 , 500 , 0 , 500 ,
   1319 , 500 , 0 , 500 , 1397 , 500 , 0 , 500 ,
   1568 , 500 , 0 , 500 , 1760 , 500 , 0 , 500 ,
   1976 , 500 , 0 , 500 , 2093 , 500 , 0 , 500 ,
      0 ,   0 ,  \\ Null terminators

: notes ( a -- ) \\ Play sequence of notes, pauses and times
   begin  dup 2@  2dup or while  ?dup if
         Hertz note  else  ms
      then  2 cells +
   repeat  drop ;
\nwnotused{scale.f}\nwendcode{}\nwbegindocs{36}\nwdocspar

Notes for the SwiftX version only:
\begin{enumerate}
\item The only purpose here is to define \texttt{\textbf{notes}} to process
the sequence of notes. The sequences of notes are not things that
need to be created within the target. Thus, even define tables are
pretty much pointless.
\item However, this did reveal that SwiftX is using timer 0 to support the
\texttt{\textbf{ms}} and \texttt{\textbf{counter ... timer}} routines.
Once \texttt{\textbf{tone-on}} is executed \texttt{\textbf{counter}}
no longer changes and \texttt{\textbf{ms}} hangs until the board is
reset.
\end{enumerate}

\subsubsection{scale.txt\index{scale.txt}}

\nwenddocs{}\nwbegincode{37}\sublabel{NWARD9-sca9-1}\nwmargintag{{\nwtagstyle{}\subpageref{NWARD9-sca9-1}}}\moddef{scale.txt~{\nwtagstyle{}\subpageref{NWARD9-sca9-1}}}\endmoddef
\\ Play musical scale                      14Apr14
chop-scale
marker chop-scale
CP @  \\ sequence of notes, pauses and times
   1047 , 500 , 0 , 500 , 1175 , 500 , 0 , 500 ,
   1319 , 500 , 0 , 500 , 1397 , 500 , 0 , 500 ,
   1568 , 500 , 0 , 500 , 1760 , 500 , 0 , 500 ,
   1976 , 500 , 0 , 500 , 2093 , 500 , 0 , 500 ,
      0 ,   0 ,  \\ Null terminators
CONSTANT scale

: notes ( a -- ) \\ Play sequence of notes, pauses and times
   begin  dup I@ >r  2 + dup I@ >r  2+
      r> r> 2dup or while  ?dup if
         Hertz note  else  ms
      then  repeat  drop ;
flush
\nwnotused{scale.txt}\nwendcode{}\nwbegindocs{38}\nwdocspar

Notes for the eForth version only:
\begin{enumerate}
\item Since \texttt{\textbf{ms}} does not use timer 0 in eForth, I just
need to get the table into eForth properly. It's not portable, but
it does work.
\end{enumerate}

\subsubsection{Tutorial6b}

\nwenddocs{}\nwbegincode{39}\sublabel{NWARD9-pit9-1}\nwmargintag{{\nwtagstyle{}\subpageref{NWARD9-pit9-1}}}\moddef{pitches.h~{\nwtagstyle{}\subpageref{NWARD9-pit9-1}}}\endmoddef
/*************************************************
 * Public Constants
 *************************************************/
#define NOTE_B0  31
#define NOTE_C1  33
#define NOTE_CS1 35
#define NOTE_D1  37
#define NOTE_DS1 39
#define NOTE_E1  41
#define NOTE_F1  44
#define NOTE_FS1 46
#define NOTE_G1  49
#define NOTE_GS1 52
#define NOTE_A1  55
#define NOTE_AS1 58
#define NOTE_B1  62
#define NOTE_C2  65
#define NOTE_CS2 69
#define NOTE_D2  73
#define NOTE_DS2 78
#define NOTE_E2  82
#define NOTE_F2  87
#define NOTE_FS2 93
#define NOTE_G2  98
#define NOTE_GS2 104
#define NOTE_A2  110
#define NOTE_AS2 117
#define NOTE_B2  123
#define NOTE_C3  131
#define NOTE_CS3 139
#define NOTE_D3  147
#define NOTE_DS3 156
#define NOTE_E3  165
#define NOTE_F3  175
#define NOTE_FS3 185
#define NOTE_G3  196
#define NOTE_GS3 208
#define NOTE_A3  220
#define NOTE_AS3 233
#define NOTE_B3  247
#define NOTE_C4  262
#define NOTE_CS4 277
#define NOTE_D4  294
#define NOTE_DS4 311
#define NOTE_E4  330
#define NOTE_F4  349
#define NOTE_FS4 370
#define NOTE_G4  392
#define NOTE_GS4 415
#define NOTE_A4  440
#define NOTE_AS4 466
#define NOTE_B4  494
#define NOTE_C5  523
#define NOTE_CS5 554
#define NOTE_D5  587
#define NOTE_DS5 622
#define NOTE_E5  659
#define NOTE_F5  698
#define NOTE_FS5 740
#define NOTE_G5  784
#define NOTE_GS5 831
#define NOTE_A5  880
#define NOTE_AS5 932
#define NOTE_B5  988
#define NOTE_C6  1047
#define NOTE_CS6 1109
#define NOTE_D6  1175
#define NOTE_DS6 1245
#define NOTE_E6  1319
#define NOTE_F6  1397
#define NOTE_FS6 1480
#define NOTE_G6  1568
#define NOTE_GS6 1661
#define NOTE_A6  1760
#define NOTE_AS6 1865
#define NOTE_B6  1976
#define NOTE_C7  2093
#define NOTE_CS7 2217
#define NOTE_D7  2349
#define NOTE_DS7 2489
#define NOTE_E7  2637
#define NOTE_F7  2794
#define NOTE_FS7 2960
#define NOTE_G7  3136
#define NOTE_GS7 3322
#define NOTE_A7  3520
#define NOTE_AS7 3729
#define NOTE_B7  3951
#define NOTE_C8  4186
#define NOTE_CS8 4435
#define NOTE_D8  4699
#define NOTE_DS8 4978
\nwnotused{pitches.h}\nwendcode{}\nwbegindocs{40}\nwdocspar
\nwenddocs{}\nwbegincode{41}\sublabel{NWARD9-TutE.2-1}\nwmargintag{{\nwtagstyle{}\subpageref{NWARD9-TutE.2-1}}}\moddef{Tutorial6b.ino~{\nwtagstyle{}\subpageref{NWARD9-TutE.2-1}}}\endmoddef
/* Tutorial 6b: Playing an Melody */
#include "pitches.h"
// notes in the melody:
int melody[] = \{    NOTE_C4, NOTE_G3,NOTE_G3, NOTE_A3, NOTE_G3,0, NOTE_B3, NOTE_C4\};
// note durations: 4 = quarter note, 8 = eighth note, etc.:
int noteDurations[] = \{4, 8, 8, 4,4,4,4,4 \};
void setup() \{
  // iterate over the notes of the melody:
  for (int thisNote = 0; thisNote < 8; thisNote++) \{
    // to calculate the note duration, take one second
    // divided by the note type.
    //e.g. quarter note = 1000 / 4, eighth note = 1000/8, etc.
    int noteDuration = 1000/noteDurations[thisNote];
    tone(8, melody[thisNote],noteDuration);
    // to distinguish the notes, set a minimum time between them.
    // the note's duration + 30% seems to work well:
    int pauseBetweenNotes = noteDuration * 1.30;
    delay(pauseBetweenNotes);
    // stop the tone playing:
    noTone(8);
  \}
\}
void loop() \{
  // no need to repeat the melody.
\}
\nwnotused{Tutorial6b.ino}\nwendcode{}\nwbegindocs{42}\nwdocspar


\subsubsection{pitches.txt\index{pitches.txt}}

\nwenddocs{}\nwbegincode{43}\sublabel{NWARD9-pitB-1}\nwmargintag{{\nwtagstyle{}\subpageref{NWARD9-pitB-1}}}\moddef{pitches.txt~{\nwtagstyle{}\subpageref{NWARD9-pitB-1}}}\endmoddef
\\ pitches.txt to Demo Musical notes      DaR 2014-04-14
\\ See: http://www.phy.mtu.edu/~suits/notefreqs.html
chop-pitches
marker chop-pitches
CP @ \\ 3 dimentional array of note frequencies
(        C      D      E      F      G      A      B )
( 0 )   16 ,   18 ,   21 ,   22 ,   24 ,   27 ,   31 ,
( # )   17 ,   19 ,    0 ,   23 ,   26 ,   29 ,    0 ,
( 1 )   33 ,   37 ,   41 ,   44 ,   49 ,   55 ,   62 ,
( # )   35 ,   39 ,    0 ,   46 ,   52 ,   58 ,    0 ,
( 2 )   65 ,   73 ,   82 ,   87 ,   98 ,  110 ,  123 ,
( # )   69 ,   78 ,    0 ,   93 ,  104 ,  117 ,    0 ,
( 3 )  131 ,  147 ,  165 ,  175 ,  196 ,  220 ,  247 ,
( # )  139 ,  156 ,    0 ,  185 ,  208 ,  233 ,    0 ,
( 4 )  262 ,  294 ,  330 ,  349 ,  392 ,  440 ,  494 ,
( # )  277 ,  311 ,    0 ,  370 ,  415 ,  466 ,    0 ,
( 5 )  523 ,  587 ,  659 ,  698 ,  784 ,  880 ,  988 ,
( # )  554 ,  622 ,    0 ,  740 ,  831 ,  932 ,    0 ,
( 6 ) 1047 , 1175 , 1319 , 1397 , 1568 , 1760 , 1976 ,
( # ) 1109 , 1245 ,    0 , 1480 , 1661 , 1865 ,    0 ,
( 7 ) 2093 , 2349 , 2637 , 2794 , 3136 , 3520 , 3951 ,
( # ) 2217 , 2489 ,    0 , 2960 , 3322 , 3729 ,    0 ,
( 8 ) 4186 , 4699 , 5274 , 5588 , 6272 ,    0 ,    0 ,
( # ) 4435 , 4978 ,    0 , 5920 ,    0 ,    0 ,    0 ,
CONSTANT pitches
flush
\nwnotused{pitches.txt}\nwendcode{}\nwbegindocs{44}\nwdocspar

Notes for the eForth version only:
\begin{enumerate}
\item There's no reason to name every note, even with defines, because we
are going to parse a Ring Tone Text Transfer Language string.
\item Once again, this is not portable and the equivalent version will not
even compile in the SwiftX evaluation version due to its size.
\item The 0 values in this 3-dimensional (note, sharp, octave) table are
invalid notes, so some error handling must be provided.
\item A sharp is the same as a flat of the next note, so only one is needed,
but the other must be calculated.
\end{enumerate}

\subsection{Tutorial 7: Counting Down with a 7 Segment LED}

\url{http://tinyurl.com/megobz3/tutorial-7-counting-down-with-a-7-segment-led/}


\subsubsection{Tutorial7}

\nwenddocs{}\nwbegincode{45}\sublabel{NWARD9-TutD.6-1}\nwmargintag{{\nwtagstyle{}\subpageref{NWARD9-TutD.6-1}}}\moddef{Tutorial7.ino~{\nwtagstyle{}\subpageref{NWARD9-TutD.6-1}}}\endmoddef
//  Tutorial 7: 7 Segment LED
//
// Define the LED digit patterns, from 0 - 9
// Note that these patterns are for common anode displays
// 0 = LED on, 1 = LED off:
// Digital pin: 2,3,4,5,6,7,8
//              a,b,c,d,e,f,g
byte seven_seg_digits[10][7] = \{ \{ 0,0,0,0,0,0,1 \},  // = 0
                                 \{ 1,0,0,1,1,1,1 \},  // = 1
                                 \{ 0,0,1,0,0,1,0 \},  // = 2
                                 \{ 0,0,0,0,1,1,0 \},  // = 3
                                 \{ 1,0,0,1,1,0,0 \},  // = 4
                                 \{ 0,1,0,0,1,0,0 \},  // = 5
                                 \{ 0,1,0,0,0,0,0 \},  // = 6
                                 \{ 0,0,0,1,1,1,1 \},  // = 7
                                 \{ 0,0,0,0,0,0,0 \},  // = 8
                                 \{ 0,0,0,1,1,0,0 \}   // = 9
                               \};
void setup() \{
  pinMode(2, OUTPUT);
  pinMode(3, OUTPUT);
  pinMode(4, OUTPUT);
  pinMode(5, OUTPUT);
  pinMode(6, OUTPUT);
  pinMode(7, OUTPUT);
  pinMode(8, OUTPUT); \}
void sevenSegWrite(byte digit) \{
  byte pin = 2;
  for (byte segCount = 0; segCount < 7; ++segCount) \{
    digitalWrite(pin, seven_seg_digits[digit][segCount]);
    ++pin;
  \}
\}
void loop() \{
  for (byte count = 10; count > 0; --count) \{
   delay(1000);
   sevenSegWrite(count - 1);
  \}
  delay(3000);
\}
\nwnotused{Tutorial7.ino}\nwendcode{}\nwbegindocs{46}\nwdocspar


\subsubsection{count.f\index{count.f}}

\nwenddocs{}\nwbegincode{47}\sublabel{NWARD9-cou7-1}\nwmargintag{{\nwtagstyle{}\subpageref{NWARD9-cou7-1}}}\moddef{count.f~{\nwtagstyle{}\subpageref{NWARD9-cou7-1}}}\endmoddef
\\ count.f to Demo 7-segment LED control      DaR 2014-04-08
\\ Define LED segment port bits and segment order
CREATE LEDS   7 2* 1 + allot \\ Number of segments, then order
: \\LEDS ( --- Initialize the RAM array )   7 LEDS C!
   LEDS count 2* PortD fill \\ Overfill the Port addresses to save code
   2 LEDS 1 + C!  3 LEDS 3 + C!  4 LEDS 5 + C!  5 LEDS 7 + C!
   6 LEDS 9 + C!  7 LEDS 11 + C!  0 LEDS 13 + C!
   PortB LEDS 14 + C! ; \\ Last segment on PortB
CREATE SEGS \\ Try flash based array on SwiftX
   0 c, 0 c, 0 c, 0 c, 0 c, 0 c, 1 c, \\ = 0
   1 c, 0 c, 0 c, 1 c, 1 c, 1 c, 1 c, \\ = 1
   0 c, 0 c, 1 c, 0 c, 0 c, 1 c, 0 c, \\ = 2
   0 c, 0 c, 0 c, 0 c, 1 c, 1 c, 0 c, \\ = 3
   1 c, 0 c, 0 c, 1 c, 1 c, 0 c, 0 c, \\ = 4
   0 c, 1 c, 0 c, 0 c, 1 c, 0 c, 0 c, \\ = 5
   0 c, 1 c, 0 c, 0 c, 0 c, 0 c, 0 c, \\ = 6
   0 c, 0 c, 0 c, 1 c, 1 c, 1 c, 1 c, \\ = 7
   0 c, 0 c, 0 c, 0 c, 0 c, 0 c, 0 c, \\ = 8
   0 c, 0 c, 0 c, 1 c, 1 c, 0 c, 0 c, \\ = 9
: sevenSegWrite ( digit --- turn on the segments for the given digit )
   LEDS count  rot over * SEGS +  -rot 0 do
      swap count >r  swap count >r count r>  r> if
         PoBiHi  else  PoBiLo
      then
   loop  2drop ;
: counts ( time cycles --- produce cycles of digits )   \\LEDS
   LEDS count 0 do  count >r count r> PoBiOut  loop  drop
   0 do  10 dup 0 do  dup i - 1- sevenSegWrite  over ms
   loop  drop  loop  drop ;
\nwnotused{count.f}\nwendcode{}\nwbegindocs{48}\nwdocspar

Notes for the SwiftX version only:
\begin{enumerate}
\item Although I used the same RAM based array as I did in the \texttt{\textbf{cycle}}
tutorial, I was able to use a ``proper'' flash based array in SwiftX.
\end{enumerate}

\section{Document Processing}

A script for converting this document to PDF form:\index{final}

\nwenddocs{}\nwbegincode{49}\sublabel{NWARD9-fin5-1}\nwmargintag{{\nwtagstyle{}\subpageref{NWARD9-fin5-1}}}\moddef{final~{\nwtagstyle{}\subpageref{NWARD9-fin5-1}}}\endmoddef
lyx -e latex $1.lyx
if [ $? == 0 ]; then
   lyx -e pdf $1.lyx
else # noweb conversion can't be called in cygwin
   lyx -e literate $1.lyx
   noweave -delay -index "$1.nw" > "$1.tex"
   pdflatex $1 latex=pdflatex fi
\nwnotused{final}\nwendcode{}\nwbegindocs{50}\nwdocspar

For marking the output with a PRELIMINARY watermark:\index{preliminary}

\nwenddocs{}\nwbegincode{51}\sublabel{NWARD9-preB-1}\nwmargintag{{\nwtagstyle{}\subpageref{NWARD9-preB-1}}}\moddef{preliminary~{\nwtagstyle{}\subpageref{NWARD9-preB-1}}}\endmoddef
echo "Make $1 PDF release notes..."
./final $1
pdftk $1.pdf stamp Preliminary.pdf output out.pdf
rm $1.pdf
mv out.pdf $1.pdf
\nwnotused{preliminary}\nwendcode{}\nwbegindocs{52}\nwdocspar

Each of these scripts can be pulled out manually given the default
{*} script defined below.

\nwenddocs{}\nwbegincode{53}\sublabel{NWARD9-*-1}\nwmargintag{{\nwtagstyle{}\subpageref{NWARD9-*-1}}}\moddef{*~{\nwtagstyle{}\subpageref{NWARD9-*-1}}}\endmoddef
echo "Extract script $2 from $1.lyx..."
rm -f $1.nw
lyx -e literate $1.lyx
notangle -t4 -R$2 $1.nw > $2
chmod a+x $2
\nwnotused{*}\nwendcode{}\nwbegindocs{54}\nwdocspar

Once that script is pulled out and named extract, the following script
can pull out all of the other scripts:\index{extract-all}

\nwenddocs{}\nwbegincode{55}\sublabel{NWARD9-extB-1}\nwmargintag{{\nwtagstyle{}\subpageref{NWARD9-extB-1}}}\moddef{extract-all~{\nwtagstyle{}\subpageref{NWARD9-extB-1}}}\endmoddef
echo "Extract all scripts..."
sedArgs="s/\\(.*\\)\\.idx:.*entry\{\\(.*\\)|hyper.*/\\1 \\2/g"
find . -type f -name \\*idx |\\
xargs grep "indexentry" |\\
sed -e "$sedArgs" |\\
xargs -n 2 ./extract
\nwnotused{extract-all}\nwendcode{}

\nwixlogsorted{c}{{*}{NWARD9-*-1}{\nwixd{NWARD9-*-1}}}%
\nwixlogsorted{c}{{Tutorial6a.ino}{NWARD9-TutE-1}{\nwixd{NWARD9-TutE-1}}}%
\nwixlogsorted{c}{{Tutorial6b.ino}{NWARD9-TutE.2-1}{\nwixd{NWARD9-TutE.2-1}}}%
\nwixlogsorted{c}{{Tutorial1.ino}{NWARD9-TutD-1}{\nwixd{NWARD9-TutD-1}}}%
\nwixlogsorted{c}{{Tutorial2.ino}{NWARD9-TutD.2-1}{\nwixd{NWARD9-TutD.2-1}}}%
\nwixlogsorted{c}{{Tutorial3.ino}{NWARD9-TutD.3-1}{\nwixd{NWARD9-TutD.3-1}}}%
\nwixlogsorted{c}{{Tutorial4.ino}{NWARD9-TutD.4-1}{\nwixd{NWARD9-TutD.4-1}}}%
\nwixlogsorted{c}{{Tutorial5.ino}{NWARD9-TutD.5-1}{\nwixd{NWARD9-TutD.5-1}}}%
\nwixlogsorted{c}{{Tutorial7.ino}{NWARD9-TutD.6-1}{\nwixd{NWARD9-TutD.6-1}}}%
\nwixlogsorted{c}{{button.f}{NWARD9-but8-1}{\nwixd{NWARD9-but8-1}}}%
\nwixlogsorted{c}{{button.txt}{NWARD9-butA-1}{\nwixd{NWARD9-butA-1}}}%
\nwixlogsorted{c}{{count.f}{NWARD9-cou7-1}{\nwixd{NWARD9-cou7-1}}}%
\nwixlogsorted{c}{{cycle.f}{NWARD9-cyc7-1}{\nwixd{NWARD9-cyc7-1}}}%
\nwixlogsorted{c}{{cycle.txt}{NWARD9-cyc9-1}{\nwixd{NWARD9-cyc9-1}}}%
\nwixlogsorted{c}{{extract-all}{NWARD9-extB-1}{\nwixd{NWARD9-extB-1}}}%
\nwixlogsorted{c}{{final}{NWARD9-fin5-1}{\nwixd{NWARD9-fin5-1}}}%
\nwixlogsorted{c}{{flasher.frt}{NWARD9-flaB.2-1}{\nwixd{NWARD9-flaB.2-1}}}%
\nwixlogsorted{c}{{flasher.txt}{NWARD9-flaB-1}{\nwixd{NWARD9-flaB-1}}}%
\nwixlogsorted{c}{{game.f}{NWARD9-gam6-1}{\nwixd{NWARD9-gam6-1}}}%
\nwixlogsorted{c}{{game.txt}{NWARD9-gam8-1}{\nwixd{NWARD9-gam8-1}}}%
\nwixlogsorted{c}{{io-core.f}{NWARD9-io*9-1}{\nwixd{NWARD9-io*9-1}}}%
\nwixlogsorted{c}{{pitches.h}{NWARD9-pit9-1}{\nwixd{NWARD9-pit9-1}}}%
\nwixlogsorted{c}{{pitches.txt}{NWARD9-pitB-1}{\nwixd{NWARD9-pitB-1}}}%
\nwixlogsorted{c}{{preliminary}{NWARD9-preB-1}{\nwixd{NWARD9-preB-1}}}%
\nwixlogsorted{c}{{scale.f}{NWARD9-sca7-1}{\nwixd{NWARD9-sca7-1}}}%
\nwixlogsorted{c}{{scale.txt}{NWARD9-sca9-1}{\nwixd{NWARD9-sca9-1}}}%
\nwixlogsorted{c}{{tone.f}{NWARD9-ton6-1}{\nwixd{NWARD9-ton6-1}}}%
\nwixlogsorted{c}{{volts.f}{NWARD9-vol7-1}{\nwixd{NWARD9-vol7-1}}}%
\nwbegindocs{56}\nwdocspar

\printindex{}
\end{document}
\nwenddocs{}
